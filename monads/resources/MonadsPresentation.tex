\documentclass{beamer}

\usepackage{listings}

\begin{document}
    \title{The Simple Monad}
    \subtitle{Solving for null pointer exceptions with monads}
    \author{Andrew Rademacher}

    \definecolor{dkgreen}{rgb}{0,0.6,0}
    \definecolor{gray}{rgb}{0.5,0.5,0.5}
    \definecolor{mauve}{rgb}{0.58,0,0.82}

    \lstset{
        frame=tb,
        basicstyle=\ttfamily,
        columns=flexible,
        breaklines=true,
        breakwhitespace=true,
        aboveskip=3mm,
        belowskip=3mm,
        showstringspaces=false,
        numbers=left,
        numberstyle=\tiny\color{gray},
        keywordstyle=\color{blue},
        commentstyle=\color{dkgreen},
        stringstyle=\color{mauve},
        tabsize=3
    }

% FRAME: TITLE
    \frame{\titlepage}

% FRAME: GOAL
    \begin{frame}
        \frametitle{Objectives}

    \end{frame}

%FRAME: A MONAD IS NOT
    \begin{frame}
        \frametitle{A Monad is \ldots}

        \begin{itemize}
            \item Not impure.
            \item Not about effects.
            \item Not about state.
            \item Not about sequencing.
            \item Not about I/O.
            \item Not dependent on laziness.
            \item Not a back-door to cheat on immutability.
        \end{itemize}

        If you think that a monad has to do with anything above,
        check yourself before you wreck yourself.
    \end{frame}

%FRAME: A Problem Statement
    \lstset{language=Java}
    \begin{frame}[fragile=singleslide]
        \frametitle{Assume for a moment.}

        \begin{columns}[c]
            \begin{column}[T]{5cm}
                \begin{lstlisting}
class Person {
    String firstName;
    String middleName;
    String lastName;
    Address homeAddress;
    Address workAddress;
}
                \end{lstlisting}
            \end{column}
            \begin{column}[T]{5cm}
                \begin{lstlisting}
class Address {
    String line1;
    String line2;
    String city;
    String state;
    int zipCode;
}
                \end{lstlisting}
            \end{column}
        \end{columns}
        * All fields are public    
    \end{frame}

    \begin{frame}[fragile=singleslide]
        \frametitle{How do you access line2?}

        \begin{lstlisting}
Person p;
String line2 = p.homeAddress.line2;
        \end{lstlisting}
    \end{frame}

\end{document}
