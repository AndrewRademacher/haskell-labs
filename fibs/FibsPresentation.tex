\documentclass{beamer}

\usepackage{listings}
\usepackage{graphicx}
\usepackage{mathtools}

\begin{document}
    \title{Demystifying Haskell}
    \subtitle{An in-depth examination of the Fibonacci sequence.}
    \author{Andrew Rademacher}

    \definecolor{dkgreen}{rgb}{0,0.6,0}
    \definecolor{gray}{rgb}{0.5,0.5,0.5}
    \definecolor{mauve}{rgb}{0.58,0,0.82}

    \lstset{frame=tb,
        language=Java,
        aboveskip=3mm,
        belowskip=3mm,
        showstringspaces=false,
        columns=flexible,
        basicstyle={\small\ttfamily},
        numbers=none,
        numberstyle=\tiny\color{gray},
        keywordstyle=\color{blue},
        commentstyle=\color{dkgreen},
        stringstyle=\color{mauve},
        breaklines=true,
        breakatwhitespace=true
        tabsize=3
    }

% FRAME:TITLE
    \frame{\titlepage}

% FRAME:FIBONACCI
    \begin{frame}
        \frametitle{The Fibonacci Sequence}

        \begin{columns}[c]
            \begin{column}[T]{5cm}
                \begin{itemize}
                    \item Sequence is infinite
                    \item Sequence is self-referencing
                    \item Values grow exponentially
                    \item Values are always positive
                \end{itemize}
            \end{column}
            \begin{column}[T]{5cm}
                \begin{figure}
                    \centering
                    \includegraphics[height=3cm]{./fibs/images/FibonacciBlocks.png}
                \end{figure}

                \begin{figure}
                    \centering
                    \begin{math}
                        \left\{1,1,2,3,5,8...\right\}
                    \end{math}
                \end{figure}
               
                \begin{figure}
                    \centering
                    \begin{math}
                        F_{n} = F_{n-1} + F_{n-2}
                    \end{math}
                \end{figure}
            \end{column}
        \end{columns}
    \end{frame}

% FRAME:JAVA
    \begin{frame}[fragile=singleslide]
        \frametitle{Traditional Java Implementation}
        
        \begin{lstlisting}
        public static BigInteger[] getFibs(int n) {
            BigInteger[] fibs = new BigInteger[n + 1];
            fibs[0] = BigInteger.valueOf(1);
            fibs[1] = BigInteger.valueOf(1);
            for (int i = 2; i <= n; i++)
                fibs[i] = fibs[i - 1].add(fibs[i - 2]);

            return fibs;
        }
        \end{lstlisting}
    \end{frame}
\end{document}
